\documentclass[9pt]{beamer}
\usetheme{Warsaw}
\usefonttheme[onlymath]{serif}

\usepackage[utf8]{inputenc}
\usepackage[polish]{babel}
\usepackage[T1]{fontenc}
\usepackage{setspace}
\usepackage{amsmath}

\newcommand{\n}{\newline}

\linespread{1}

\title{Projekt 1, Zadanie 23}
\author{Wiktor Murawski, 333255, grupa 3, środa 12:15}
\date{}

\begin{document}

    \begin{frame}
        \frametitle{\insertauthor,\space\inserttitle}
        \begin{spacing}{2}
        Obliczanie całek $ \iint\limits_D f(x,y) \, dxdy $ na obszarze
        $  D = \{(x,y) \in \mathbb{R}^2 : |x| + |y| \leq 1\} $
        poprzez podział obszaru $ D $ na $ 4n^2 $ trójkątów przystających oraz
        zastosowanie na każdym z nich kwadratury rzędu drugiego.
        \end{spacing}
    \end{frame}

	\section{Testy Poprawności}
	\begin{frame}
		\frametitle{Wyznaczenie analityczne całki z funkcji stopnia 1}

        \begin{spacing}{2}
        Obliczymy analitycznie $ I = \iint\limits_D f(x,y) \, dx dy $ gdzie
        $ f(x,y) = ax + by + c \qquad a,b,c \in \mathbb{R}$ \n
        Niech $ D_1 = \{(x,y) \in D : x \leq 0\} $ oraz $ D_2 = \{(x,y) \in D : x > 0\} $ \n
        Oznaczmy $ I_1 = \iint\limits_{D_1} f(x,y) \, dx dy $, $ I_2 = \iint\limits_{D_2} f(x,y) \, dx dy $ \n
        Wtedy $ D = D_1 \cup D_2 $ oraz $ I = I_1 + I_2 $ \n
        $ I_1 = \int\limits_{-1}^{0}\int\limits_{-x-1}^{x+1} ax+by+c \, dy dx $ \qquad\n
        $ I_2 = \int\limits_{0}^{1}\int\limits_{x-1}^{-x+1} ax+by+c \, dy dx $ \n
        \end{spacing}

	\end{frame}


    \begin{frame}
        \frametitle{Wyznaczenie analityczne całki z funkcji stopnia 1}
        \begin{columns}
            \begin{column}{0.6\textwidth}
                % Lewa kolumna
                \begin{align*}
                    I_1 &= \int\limits_{-1}^{0}\int\limits_{-x-1}^{x+1} ax+by+c \,dydx \\
                    I_1 &= \int\limits_{-1}^{0}\left[axy + \frac{by^2}{2} + cy\right]_{-x-1}^{x+1} \,dx \\
                    I_1 &= \int\limits_{-1}^{0} 2ax^2 + 2ax + 2cx + 2c \,dx \\
                    I_1 &= 2\left[ \frac{ax^3}{3} + \frac{ax^2}{2} + \frac{cx^2}{2} + cx \right]_{-1}^{0} \\
                    I_1 &= -\frac{a}{3} + c \\
                \end{align*}
            \end{column}
            \begin{column}{0.6\textwidth}
                % Prawa kolumna
                \begin{align*}
                    I_2 &= \int\limits_{0}^{1}\int\limits_{x-1}^{-x+1} ax+by+c \,dydx \\
                    I_2 &= \int\limits_{0}^{1}\left[axy + \frac{by^2}{2} + cy\right]_{x-1}^{-x+1} \,dx \\
                    I_2 &= \int\limits_{0}^{1} - 2ax^2 + 2ax - 2cx + 2c \,dx \\
                    I_2 &= 2\left[ -\frac{ax^3}{3} + \frac{ax^2}{2} - \frac{cx^2}{2} + cx \right]_{0}^{1} \\
                    I_2 &= \frac{a}{3} + c \\
                \end{align*}
            \end{column}
        \end{columns}

        \begin{center}
            Ostatecznie otrzymujemy $ I = I_1 + I_2 = 2c $
        \end{center}

    \end{frame}

\end{document}
